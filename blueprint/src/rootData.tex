\chapter{Root data}
\label{cha:construction-root-systems}
\section{Definitions}
\begin{definition}
    \label{def:root-datum}
    \leanok
    \uses{def:root-pairing}
     A root pairing $(R,M,N,\mathcal{L},I,\alpha, \alpha^\vee,s)$ is called a {\it root datum} if 
    i) $R = \mathbb{Z}$; and ii) $M,N$ are finitely generated free $\mathbb{Z}$-modules (i.e. free abelian groups).
    
    Moreover, it is called {\it reduced} if, for each $\alpha \in \Phi$, we have 
    $\mathbb{Q}\alpha = \{\pm \alpha\} \subset M_{\mathbb{Q}}$.
\end{definition}

{\it Lean implementation note: A root datum is defined in Mathlib as an abbreviation of 
a root pairing with $R = \mathbb{Z}$ and $M,N$ finitely generated free $\mathbb{Z}$-modules. Here 
recall that the syntax for defining an abbreviation is ``abbrev Name : Type := Definition"}

For the following, we diverge from Mathlib's current terminology, preferring the definition in 
by B. Conrad. 
\begin{definition}
    \label{def:rp-semisimple}
    \leanok
    \uses{def:root-datum}
    A reduced root datum $(\mathbb{Z},M,N,\mathcal{L},I,\alpha, \alpha^\vee,s)$ is called {\it semisimple} if
    \[
        {\rm span}_{\mathbb{Q}} \Phi = M_{\mathbb{Q}} \quad \text{and} \quad {\rm span}_{\mathbb{Q}} \Phi^\vee = N_{\mathbb{Q}}.
    \] 
\end{definition}

\begin{lemma}
    If a root datum is reduced, then so is its dual. 
\end{lemma}
\begin{proof}
    
\end{proof}

\section{Constructing root data}
In this section, we construct reduced semisimple root data. We begin with a 
key lemma. 

\begin{lemma}
    Construction lemma; to be added
\end{lemma}




\begin{example}
Denote by $(\varepsilon_1, \varepsilon_2)$ the standard basis 
of $M = \mathbb{Z}^2$, 
with dual basis $(\varepsilon_1^*, \varepsilon_2^*)$ of $N = \mathbb{Z}^2$. The sets 
\[
    \Phi = \{\pm \varepsilon_1, \pm \varepsilon_2, \pm\varepsilon_1\pm\varepsilon_2\} \quad \text{and} \quad
    \Phi^\vee = \{\pm 2\varepsilon^*_1, \pm 2 \varepsilon^*_2, \pm \varepsilon_1^* \pm \varepsilon_2^* \},
\]
give rise to a reduced semisimple root datum. Interchanging the roles of $\Phi$ and $\Phi^\vee$ gives rise to another root datum.
\end{example}


