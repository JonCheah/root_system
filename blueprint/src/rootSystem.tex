\section{Finite root systems}
\label{sec:finite-root-systems}

Throughout, we fix $k$ to be a field of characteristic $0$. We follow the treatement given in Bourbaki, 
with no claims of originality.

\begin{definition}
    Let $V$ be a vector space over $k$. A linear map $s : V \to V$ is called a {\it pseudo-reflection} if 
    the rank of ${\rm Id}-s$ is $1$. 
\end{definition}
Denote by $D$ the image of ${\rm Id}-s$. By definition, $D$ is a one-dimensional vector subspace of $V$. 
Thus, given $a \in D$ with $a \neq 0$, there exists some $a^* \in V^*$ such that $({\rm Id}-s)(x)= a^*(x) a$ for 
all $x \in V$. Moreover, $ker(a^*) = D$.

\begin{definition}
    Let $s$ be a pseudo-reflection in $V$. Then, $s$ is called a {\it reflection} if $s^2 = {\rm Id}$.
\end{definition}

\begin{lemma}
    Let $V$ be a vector space over $k$, $s$ a reflection in $V$. Then, 
    \[
        V = ker({\rm Id}-s) \oplus ker(s+{\rm Id}).
    \]
\end{lemma}

\begin{definition}
  \label{def:root-system}
  Let $V$ be a vector space over $k$, $\Phi$ a subset of $V$. We say that $\Phi$ is a 
  \emph{root system} in $V$ if the following conditions are satisfied:
    \begin{enumerate}
        \item $\Phi$ is finite, does not contain the zero vector and spans $V$.
        \item For every $\alpha$ in $\Phi$, there exists an element $\alpha^\vee \in V^*$ such that 
            $\alpha^\vee(\alpha) = 2$, and the reflection $s_{\alpha,\alpha^\vee}: V \to V$ fixes $\Phi$.
        \item (Crystallographic condition) For every $\alpha\in \Phi$, we have $\alpha^\vee(\Phi) \subset \mathbb{Z}$
    \end{enumerate}
\end{definition}

\begin{lemma}
   Let $\Phi$ be a root system in $V$, $\alpha \in \Phi$. If there exist $\alpha_i^\vee,i=1,2 \in V^*$ such that
    $\alpha_i^\vee(\alpha) =  2, i=1,2$ and $s_{\alpha_i^\vee,\alpha}(\Phi) \subset \Phi$,
     then $\alpha_1^\vee = \alpha_2^\vee$.
\end{lemma}
\begin{proof}

\end{proof}
